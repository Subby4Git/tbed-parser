\documentclass{article}
\usepackage{interspeech2008,amssymb,amsmath,epsfig}
\usepackage{epsfig}
\setcounter{page}{1}
\sloppy   % better line breaks
\ninept
%SM below a registered trademark definition
\def\reg{{\rm\ooalign{\hfil
     \raise.07ex\hbox{\scriptsize R}\hfil\crcr\mathhexbox20D}}}

%% \newcommand{\reg}{\textsuperscript{\textcircled{\textsc r}}}

\newcommand{\abs}[1]{\left\vert#1\right\vert}
\newcommand{\set}[1]{\left\{#1\right\}}
\newcommand{\citet}[1]{\cite{#1}}
\newcommand{\citep}[1]{\cite{#1}}
\newcommand{\ut}{\textit}
\newcommand{\lit}[1]{\footnote{Lit. translation: #1}}

\newcommand{\efgr}[2]{
  \begin{figure}[htbp]
    \makebox[8.5cm]{\framebox[5cm]{\rule{0pt}{5cm}}}
    \caption{#2}
    \label{#1}
  \end{figure}
}

\newcommand{\fgrparam}[4]{
  \begin{figure}[htbp]
    \begin{center}
      \leavevmode
      \includegraphics[#1]{#2}
    \end{center}
    \vspace{-0.5cm}
    \caption{#4}
    \label{#3}
  \end{figure}
}


\title{Transformation-based Learning for Semantic parsing}

%%%%%%%%%%%%%%%%%%%%%%%%%%%%%%%%%%%%%%%%%%%%%%%%%%%%%%%%%%%%%%%%%%%%%%%%%%
%% If multiple authors, uncomment and edit the lines shown below.       %%
%% Note that each line must be emphasized {\em } by itself.             %%
%% (by Stephen Martucci, author of spconf.sty).                         %%
%%%%%%%%%%%%%%%%%%%%%%%%%%%%%%%%%%%%%%%%%%%%%%%%%%%%%%%%%%%%%%%%%%%%%%%%%%
%\makeatletter
%\def\name#1{\gdef\@name{#1\\}}
%\makeatother
%\name{{\em Firstname1 Lastname1, Firstname2 Lastname2, Firstname3 Lastname3,}\\
%      {\em Firstname4 Lastname4, Firstname5 Lastname5, Firstname6 Lastname6,
%      Firstname7 Lastname7}}
%%%%%%%%%%%%%%% End of required multiple authors changes %%%%%%%%%%%%%%%%%

\makeatletter
\def\name#1{\gdef\@name{#1\\}}
\makeatother
\name{{\em F. Jur\v{c}\'{i}\v{c}ek, M. Ga\v{s}i\'{c}, S. Keizer, F. Mairesse, B. Thomson, K. Yu, and S. Young}}

% \address{$^1$Department of Speech and Hearing, University of Marsupials,
%   Voiceland, KoalaCountry \\
%   $^2$Department of Linguistics, University of Speechcity, Speechland\\
% {\small \tt jeg@sci.voice.edu, DBark@ling.speech.edu}}

\address{Engineering Department, Cambridge University, CB2 1PZ, UK \\
\\
{\small \tt \{fj228, mg436, sk561, f.mairesse, brmt2, ky219, sjy\}@eng.cam.ac.uk}
}

%\twoauthors{John E. Grunt Jr.}{Department of Speech and Hearing \\
%  University of Marsupials, Voiceland, KoalaCountry \\
%  {\small \tt jeg@sci.voice.edu} }
%  {Donald Bark}{Department of Linguistics \\
%  University of Speechcity, Speechland \\
%  {\small \tt DBark@ling.speech.edu} }

%
\begin{document}
\maketitle
%
\begin{abstract}
This paper presents a semantic parser that transforms an initial semantic hypothesis into the correct semantics by applying an ordered list of transformation rules. These rules are learnt automatically from a training corpus with no prior linguistic knowledge and no alignment between words and semantic concepts. The learning algorithm produces a compact set of rules which enables the parser to be very efficient while retaining high accuracy. We show that this parser is competitive with respect to the state-of-the-art semantic parsers on the ATIS and TownInfo tasks.
\end{abstract}
\vspace{0.1cm}
\noindent{\bf Index Terms}: spoken language understanding, semantics, natural language processing, transformation-based learning

%%%%%%%%%%%%%%%%%%%%%%%%%%%%%%%%%%%%%%%%%%%%%%%%%%%%%%%%%%%%%%%%%%%%%
% \textbf{Check how you use input, output, hypothesis semantics!}

\section{Introduction}
% Semantic parsing is an important part of a spoken dialogue system \cite{williams07}. 
The goal of semantic parsing is to map natural language to a formal meaning representation - semantics. Such semantics can be either defined by a grammar, e.g. LR grammar for the GeoQuery domain \cite{kate05}, or by frames and slots, e.g. the TownInfo domain \cite{mairesse09}. Table \ref{tbl:sem:example} shows an example of the frame and slot semantics from the ATIS dataset \cite{atis94}. Each frame has a goal and a set of slots. Each slot is composed of a slot name, e.g. ``from.city'',
% , a slot equal sign, e.g. ``='' or ``!='', 
and a slot value, e.g. ``Washington''. As dialogue managers commonly use semantics in the form of frames and slots \cite{williams07,thomson08}, our approach learns to map directly from natural language into frame and slot semantics.

% The semantics, which is designed by a domain expert, is directly executable by a question answering system \cite{wong06} or by a dialogue manager in a spoken dialogue system.

A dialogue system needs a semantic parser which is accurate and robust, easy to build, and fast. This paper presents a parsing technique which provides state-of-the-art performance and robustness to ill formed utterances. The parser does not need any handcrafted linguistic knowledge and it can learn from data which is not semantically annotated at the word-level. Finally, it learns a compact set of rules that allow it to perform real-time semantic parsing. Note that modern statistical dialogue systems typically exploit multiple ASR hypotheses. Hence, the semantic parser has to process an N-best list of user utterances every turn where N$\sim$10 to 100.

% can be understood as machine translation from a natural language to a formal language. First, we do not not have formal grammar for natural language is ungrammatical, include hesitations, and very often only fragments of complete utterances, e.g. ``Boston to Miami tomorrow''. 

In our approach, we adapt Transformation-Based Learning (TBL) \cite{brill95} to the problem of semantic parsing. We attempt to find an ordered list of transformation rules which iteratively improve the initial semantic annotation. 
% During parsing, transformation rules are sequentially applied and
In each iteration, a transformation rule corrects some of remaining errors in the semantics.
To handle long-range dependencies between words, we experiment with features extracted from dependency parse trees provided by the RASP syntactic parser \cite{rasp06}.

% A transformation parsing assumes:
% \begin{itemize}
% \item Initial annotation is obtained by a initial annotator.
% \item Transformations are applied in sequence. 
% \item Transformations progressively change from general to specific.
% \item Results of previous transformations are visible to following transformations.
% \item Transformations correct errors of the previous transformations.
% \end{itemize} 

In the next section, we describe previous work on mapping natural language into a formal meaning representation. Section \ref{sec:tbl} presents an example of TBL semantic parsing, discusses templates for transformation rules, and describes the learning process. Section \ref{sec:evaluation} compares the TBL parser to the previously developed semantic parsers on the ATIS \cite{atis94} and TownInfo \cite{mairesse09} domains. Finally, Section \ref{sec:conlusion} concludes this work.

\begin{table}
\begin{center}
\begin{tabular}{lll} 
%   \hline
  \multicolumn{3}{l}{what are the lowest airfare from Washington DC to Boston} \\
  \hline
  GOAL          & = & airfare \\
  airfare.type  & = & lowest \\
  from.city     & = & Washington \\
  from.state    & = & DC \\
  to.city       & = & Boston \\
%   \hline
\end{tabular} 
\end{center}
\vspace{-0.5cm}
\caption{Example of frame and slot semantics from the ATIS dataset \cite{atis94}.}
\label{tbl:sem:example}
\end{table}

\section{Related work}

In Section \ref{sec:evaluation}, we compare the performance of our method with four existing systems that were evaluated on the same dataset. 
First, the Hidden Vector State (HVS) technique has been used to model an approximation of a pushdown automaton with semantic concepts as non-terminal symbols \cite{he05,jurcicek08}. 
% From the output parse trees, a deterministic algorithm was used to recover slot names and their values.
Second, a Probabilistic parser using Combinatory Categorical Grammar   (PCCG) has been used to map utterances to lambda-calculus \cite{zettlemoyer07}. 
% The combinatory categorical grammar is generalized into probabilistic model by learning log-linear model. An online learning algorithm updates weights of features representing a parse tree of an input utterance. 
This technique produces state-of-the-art performance on the ATIS dataset. However, apart from using the lexical categories (city names, airport names, etc) readily available from the ATIS corpus, this method also needs a considerable number of handcrafted entries in its initial lexicon. 
Third, Markov Logic Networks (MLN) have been used to extract slot values by combining probabilistic graphical models and first-order logic \cite{meza08b}. In this approach, weights are attached to first-order clauses which represent the relationship between slot names and their values. Such weighted clauses are used as templates for features of Markov networks.
Finally, Semantic Tuple Classifiers (STC) based on support vector machines have been used to build semantic trees by recursively calling classifiers that predict fragments of the semantic representation from n-gram features \cite{mairesse09}.

In addition to the above, there is a large amount of research that is related but not directly comparable because of different in corpora or meaning representation. 
% Machine translation techniques \cite{wong06} have been used with a syntax-based translation model based on the synchronous context-free grammars. 
% Inductive logic programming \cite{tang01} have been used to incrementally develop a theory including a set of predicates. 
% In each iteration, the predicates were generalized from predicates in the theory and predicates automatically constructed from examples. 
For example, transformation techniques have been previously used to sequentially rewrite an utterance into semantics \cite{kate05}. However, our approach differs in the way the semantics is constructed. Instead of rewriting an utterance, we transform an initial semantic hypothesis. As a result, the words in the utterance can be used several times to trigger transformations of the semantics. 
% This extends our ability to handle non-compositionality phenomena in spoken language.

\section{Transformation-based parsing} \label{sec:tbl}
% This section describes the transformation-based parser. 

The TBL parser transforms an initial semantic hypothesis into the correct semantics by applying transformations from a list of rules. Each rule is composed of a trigger and a transformation. The trigger is matched against both the utterance and the semantic hypothesis, and when successfully matched, the transformation is applied to the current hypothesis. 

In the TBL parser, a trigger contains one or more conditions as follows: the utterance contains N-gram N, the goal equals G, and the semantics contains slot S. If a trigger contains more than one condition, then all conditions must be satisfied. N-gram triggers can be unigrams, bigrams, trigrams or skipping bigrams
% \footnote{In a skipping bigram, one, two or three words are skipped between two words.}
which can skip up to 3 words.
A transformation performs one of the following operations: replace the goal, add a slot, delete a slot, and replace a slot. A replacement transformation can replace a whole slot, a slot name, 
% a slot equal sign, 
or a slot value. 

Some example rules with triggers composed of unigrams, skipping bigrams, and goal matching are:

\vspace{.15cm}
\begin{tabular}{ll}
  trigger & transformation \\
  \hline 
  ``tickets''        & replace the goal by ``airfare''\\
  ``flights * from'' & replace the goal by ``flight'' \\
  \& GOAL=airfare    & \\
  ``Seattle''        & add the slot ``to.city=Seattle'' \\
  ``connecting''     & replace the slot \\
                     & ``to.city=*'' by ``stop.city=*'' \\
\end{tabular} 
\vspace{.15cm}

The first rule replaces the goal by ``airfare'' if the word ``tickets'' is in the utterance. The second rule changes the goal from ``airfare'' to ``flight'' if the utterance contains the words ``flights'' and ``from'', which can be up to 3 words apart. The fourth rule adds the slot ``to.city=Seattle'' whenever the utterance contains the word ``Seattle''. Finally, every slot name ``to.city'' is replaced by "stop.city`` if the utterance includes the word ''connecting``.

% \subsection{Rule templates}
% In the previous sections, we presented several rules which add, substitute, or delete slots. 
% The templates define what types of rules can be used for parsing. 
% A trigger 
% controls when a transformation of a semantic hypothesis can be performed and it can 
% tries to match an input utterance, an output semantics, or both. 
% For example, ('arrive',*,'Boston') is a skipping bigram which skips one word.


In the next section, we give an example of how the parsing algorithm works. Then, we detail locality constraints on the transformation rules.
Next, we describe features capturing long-range dependencies.
Finally, the automatic learning process is described. 

% These rules were selected by a learning algorithm from a large set of potential rules. Such rules are generated from a set of templates for triggers and transformations. 

\subsection{Example of Parsing} \label{sec:example}

% The parsing consists of three steps: 
% \begin{enumerate}
%   \item initial semantics is assigned as hypothesis
%   \item sequentially apply all rules\footnote{Input utterance is not modified by rules. As a result, words from the utterance can be triggered several different transformations.}
%   \item output hypothesis semantics
% \end{enumerate}
This section demonstrates the parsing process on the example: \textit{``find all the flights between Toronto and San Diego that arrive on Saturday''} 

First, the goal ``flight'' with no slots is used as the initial semantics because it is the most common goal in the ATIS dataset. As a result, the initial semantics is as follows:

\vspace{.15cm}
\begin{tabular}{lll}
  GOAL & = & flight
\end{tabular} 
\vspace{.15cm}

Second, the rules, whose triggers match the utterance and the hypothesised semantics, are sequentially applied. 
% Generally, the rules can add, delete, or substitute slots. 
% However, in this example the matching rules are only those which add slots.

\vspace{.15cm}
\begin{tabular}{ll}
  trigger & transformation \\
  \hline 
  ``between toronto''     & add the slot ``from.city=Toronto'' \\
  ``and san diego''       & add the slot ``to.city=San Diego'' \\
  ``saturday''            & add the slot ``departure.day=Saturday'' \\
\end{tabular} 
\vspace{.15cm}

After applying the transformations, we obtain the following semantic hypothesis: 

\vspace{.15cm}
\begin{tabular}{lll}
  GOAL          & = & flight \\
  from.city     & = & Toronto \\
  to.city       & = & San Diego \\
  departure.day & = & Saturday \\
\end{tabular} 
\vspace{.15cm}

% The trigger ``and Sand Diego'' is example of non-compositionality, in which the words in an utterance do not have a one-to-one correspondence with the slots in the semantics. The word ``and'' indicates that the city ``San Diego'' is slot value of the slot ``to.city''. 

% As the TBL method tends to learn and apply general rules first, 

As the date and time values are associated with the ``departure.*'' slots most of the time in the ATIS dataset,  
the parser learns to associate them with the ``departure.*'' slots. The incorrect classification of the word ``Saturday'' is a result of such a generalisation. 
However, the TBL method learns to correct its errors. Therefore, the parser also applies the error correcting rules at a later stage. For example, the following rule corrects the slot name of the slot value ``Saturday''.

\vspace{.15cm}
\begin{tabular}{ll}
  trigger & transformation \\
  \hline 
  ``arrive''            & replace the slot ``departure.day=*'' \\
                        & by ``arrival.day=*'' \\
\end{tabular} 
\vspace{.15cm}

In this case, we substitute the slot name with the correct name, to produce the following semantic hypothesis:

\vspace{.15cm}
\begin{tabular}{lll}
  GOAL          & = & flight \\
  from.city     & = & Toronto \\
  to.city       & = & San Diego \\
  arrival.day   & = & Saturday \\
\end{tabular} 

\subsection{Locality constraints} \label{sec:locality:constrain}
So far the relationship between slots and their lexical realisation has not been considered. For example, before we replace the slot ``departure.day'' by ``arrival.day'', we should test whether the word ``arrive'' is near the slot's lexical realisation. Otherwise we may accidentally trigger the substitution of the slot ``from.city=Toronto'' by ``to.city=Toronto''. This could happen if the parser had also learnt the following rule:

\vspace{.15cm}
\begin{tabular}{ll}
  trigger & transformation \\
  \hline 
  ``arrive''            & replace the slot \\
                        & ``from.city=*'' by ``to.city=*'' \\
\end{tabular} 
\vspace{.15cm}
\fgrparam{width=8cm}{./fig/words-slots-alignment.pdf}{fig:alignment}{Alignment between the words and the slots in the example utterance.}

% \textbf{XXX: I mentioned delete transformation but never defined.}

One way to handle this problem is to constrain triggers performing substitutions to be activated only if they are near the slot's lexical realisation. To do this; we track the words from the utterance that were used in triggers. Every time we apply a transformation of a slot, we store links between the words which triggered the transformation and the target slot. Such links are referred to as ``direct alignment''. 

% For the delete transformation no tracking information is kept because a slot is removed from hypothesis and never used again.

In Figure \ref{fig:alignment} (a), we see the alignment between the words and the slots in the example utterance after applying the first set of rules. The full arrows denote direct alignment created by the add-transformations. Because no rules were triggered by the words ``find all the flights'' and ``that arrive on'', those words could not be aligned directly to any of the slots. Therefore, we have to infer an appropriate alignment (see Figure \ref{fig:alignment} (a) dashed arrows). A word is aligned to a slot if the alignment does not cross any direct alignment.
% To compute derived alignment, first we order the slots so that the slot aligned with the left-most word is the first and the ordering results in minimum instances of direct alignment crossing. Then, every unaligned word is aligned with the nearest left and the right slot. 
% The dashed lines denote derived alignment - alignment computed from the direct alignment. 
In Figure \ref{fig:alignment} (a), the phrase ``find all the flights'' can be aligned to the slot ``from.city=Toronto'' only (dashed arrows). The phrase ``that arrive on'' can be aligned to two slots ``to.city=San Diego'' and ``departure.day=Saturday''. 

In Figure \ref{fig:alignment} (a), we see that the rule given in Section \ref{sec:example} meets the locality constraint. We can see the alignment of the phrase ``that arrive on'' has changed as a result of applying it. First, the word ``arrive'' is aligned to the slot ``arrival.day=Saturday''. Second, the word ``on'' must be aligned to the same slot as the word ``arrive''. There is no change in the alignment of the word ``that''.

\subsection{Improving the disambiguation of long-range dependencies}
\label{sec:dep:trees}

\fgrparam{width=6cm}{./fig/dep-tree.pdf}{fig:dep:tree}{Dependency tree of the utterance ''show the cheapest flights from Boston to Miami arriving before 7pm on Monday``.}

Besides simple n-grams and skipping bigrams, more complex lexical features can be used. Kate \citep{kate08} used manually annotated dependency trees to capture long-range relationships between words. In a dependency tree, each word is viewed as the dependant of one other word, with the exception of the root. Dependency links represent grammatical relationships between words.
Kate showed that word dependencies significantly improve semantic parsing because long-range dependencies from an utterance tend to be local in a dependency tree. For example, the words ''arriving`` and ''Monday`` are neighbours in the dependency tree but they are four words apart in the utterance (see Figure \ref{fig:dep:tree}).

Instead of using manually annotated word dependencies \cite{kate08}, we used dependencies provided by the RASP dependency parser \cite{rasp06}. 
% First, some pre-processing had to be done so that the RASP parser was able to accurately parse utterances with unknown named entities such as ''New York``. 
% First of all, we had to add capitalization and punctuation into the ATIS data to be able to use the RASP parser. The RASP parser without proper capitalization fails to tag ''new`` and ''york`` as NP and instead of this it tags ''new`` as ''JJ`` and 'york' as NP and the dependencies generated by the parser are unsatisfactory. 
New n-gram features were generated in which a word history is given by links between words. For example, the algorithm would generate bi-gram ('arriving','Monday') for the word ''Monday``.
% Even though the dependencies generated the RASP parser are not perfect, the new features increase performance in F-measure on ATIS data. 
Note however that RASP was used ''off-the-shelf`` and more accurate dependencies could be obtained by adapting it to the target domain.

% Secondly, we generated long-range features by using POS tags\footnote{We used POS tags provided by the RASP parser; however, any POS tagger can be used instead.}. 
% Our motivation was work of \cite{meza08b} who handcrafted features using words ''arrive``, ''arriving``, ''leave``, and ''leaving``. These handcrafted features disambiguate large number of semantic parsing errors in ATIS data because large portion or errors is caused by confusions between concepts ''arrival.time`` and ''departure.time``, ''arrival.day`` and ''departure.day``, etc. We generalized this approach and we automatically find features which  disambiguate semantics of words like ''Monday``, ''7pm``, and ''Boston``. As a result, we generate a new type of bigrams for a word and the nearest verb, preposition, etc. \textbf{We use all parts-of-speech provided by RASP and the learning algorithm chooses the most discriminative features. Among those learned are not only the words used by Meza-Rui but also words like ''stop``, ''reach``, ''buy`` and prepositions like ''at``, ''from``, ''to``, etc.} For example, for the nearest verb for the word ''Morning`` in the utterance from Figure \ref{fig:dep:tree} is ''arrive`` and such bigram would look be written as ('arrive',VV,'Monday') where VV stands for verb. This features assumes that the left-to-right tendency is dominant and the words in vicinity of lexical realization of a slot value affect the meaning the most.

\subsection{Learning} \label{sec:tbl:learning}
The main idea behind transformation-based learning \cite{brill95} is to learn an ordered list of rules which incrementally improve an initial semantic hypotheses (see the algorithm in Figure~\ref{alg:tbl:learning})\footnote{The list of rules must be ordered because each learnt rule corrects the errors remaining after applying the preceding rules.}. The initial assignment is made based on simple statistics - the most common goal is used as initial semantics. The learning is conducted in a greedy fashion, and at each step the algorithm chooses the transformation rule that reduces the largest number of errors in hypotheses. Errors include goal substitutions, slot insertions, slot deletions, and slot substitutions. The learning process stops when the algorithm cannot find a rule that improves the hypotheses beyond some pre-set threshold. Note that no prir alignment between words and semantic concepts is needed.

% To limit overfitting the training data, we prune some rules which are learned at the end of the learning. We sequentially apply each rule on the development set and we measure the number of errors. At the end, we chose the N first rules for which the parser gets the lowest number of errors.

\begin{figure}
\begin{footnotesize}\textsc{
\begin{enumerate}
  \item assign initial semantics to each utterance
  \item repeat as long as the number of errors on the training set decreases
  \begin{enumerate}
    \item generate all rules which correct at least one error in the training set
    \item measure the number of corrected errors by each rule
    \item select the rule with the largest number of corrected errors
    \item apply the selected rule to the current state of the training set
    \item stop if the number of corrected errors is smaller than threshold T.
  \end{enumerate}
\end{enumerate}}
\end{footnotesize}
\vspace{-0.25cm}
\caption{Rule learning algorithm.}
\label{alg:tbl:learning}
\end{figure} 

As in the previous work \cite{mairesse09,he05,zettlemoyer07,meza08b}, we make use of a database with lexical realisations of some slots, e.g. city and airport names. Since the number of possible slot values for each slot is usually very high, the use of a database results in a more robust parser. In our method, we replace lexical realisations of slot values with category labels before parsing, e.g. ``i want to fly from CITY''. 
% Similarly, we replace slot values in the semantics.
After parsing we use a deterministic algorithm to recover the original values for category labels, which is detailed in \cite{mairesse09}.

% To speed up the training process, we select multiple best performing rules and the performance of worst selected rule has to be at least at least 80\% of the best rule. We found that selection of multiple rules during learning does not affect the performance of the parser and at the same time it decrease the learning time.

% During the decoding phase, the test set is initialized with the same initial class's assignment. Each rule is than applied, in the order it was learned, to the test set. The final classification is the one attained when all rules have been applied.

\section{Evaluation} \label{sec:evaluation}

In this section, we evaluate our parser on two distinct corpora, and compare our results with state-of-the-art techniques and a handcrafted Phoenix parser \cite{ward91}. 

\subsection{Datasets} \label{sec:dataset}

In order to compare our results with previous work \cite{mairesse09, he05,zettlemoyer07,meza08b},
we apply our method to the ATIS  dataset \cite{atis94}. 
% This dataset consists of user requests for flight information, for example ``find flight from San Diego to Phoenix on Monday''. 
We use 5012 utterances for training, and the DEC94 dataset as development data. As in previous work, we test our method on the 448 utterances of the NOV93 dataset, and the evaluation criterion is the F-measure of the number of reference slot/value pairs that appear in the output semantics (e.g., from.city = New York). He \& Young detail the test data extraction process in \cite{he05}.

Our second dataset consists of tourist information dialogues in a fictitious town (TownInfo). The dialogues were collected through user trials in which users searched for information about a specific venue by interacting with a dialogue system in a noisy background. 
% For example, the utterance ``I would like a Chinese restaurant'' is represented as
% \vspace{.15cm}
% \begin{tabular}{lll}
%   GOAL       & = & inform \\
%   food       & = & Chinese \\
%   type       & = & restaurant \\
% \end{tabular} 
% \vspace{.15cm}
The TownInfo training, development, and test sets respectively contain 8396, 986 and 1023 transcribed utterances.  The data includes the transcription of the top hypothesis of a speech recogniser, which allows us to evaluate the robustness of our models to recognition errors (word error rate = 34.4\%). 
We compare our model with the STC parser \cite{mairesse09} and the handcrafted Phoenix parser \cite{ward91}. The Phoenix parser implements a partial matching algorithm that was designed for robust spoken language understanding.

\subsection{Results}

\begin{table}
\begin{center}
\begin{tabular}{|l|ccc|}
\hline \makebox[2.99cm]{\bf Parser} & \makebox[1.1cm]{\bf Prec} & \makebox[1.1cm]{\bf Rec} & \bf F \\ \hline 
\multicolumn{4}{|l|}{\textbf{ATIS dataset with transcribed utterances:}} \\
\hline
TBL   & 96.37 & 95.12 & 95.74 \\
PCCG  & 95.11 & 96.71 & 95.9 \\
STC   & 96.73 & 92.37 & 94.50 \\
HVS   & - & - & 90.3  \\
MLN   & - & - & 92.99 \\
\hline
\multicolumn{4}{|l|}{\textbf{TownInfo dataset with transcribed utterances:}} \\
\hline
TBL      & 96.05 & 94.66 & 95.35 \\
STC      & 97.39 & 94.05 & 95.69 \\
Phoenix  & 96.33 & 94.22 & 95.26 \\
\hline
\multicolumn{4}{|l|}{\textbf{TownInfo dataset with ASR output:}} \\
\hline
TBL      & 92.72 & 83.42 & 87.82 \\
STC      & 94.03 & 83.73 & 88.58 \\
Phoenix  & 90.28 & 79.49 & 84.54 \\
\hline
\end{tabular}
\end{center}
\vspace{-0.5cm}
\caption{Slot/value precision (Prec), recall (Rec) and F-measure (F) for the ATIS and TownInfo datasets. 
% TBL parser is compared with Phoenix parser and STC classifier \cite{mairesse09} on the TownInfo dataset and compared with HVS parser \cite{he05}, MLN parser \cite{meza08b}, STC classifier, and PCCG parser \cite{zettlemoyer07} on the ATIS dataset.
}
\label{tbl:results-final} 
\end{table}

The results for both datasets are shown in Table \ref{tbl:results-final}.
The model accuracy is measured in terms of precision, recall, and F-measure (harmonic mean of precision and recall) of the slot/value pairs. Both slot and value must be correct to count as a correct classification.

Results on the ATIS dataset show that the TBL parser (F-measure = 95.74\%) is competitive with respect to the Zettlemoyer \& Collins' PCCG model \cite{zettlemoyer07} (95.9\%). Note that this PCCG model makes use of a considerably large number of handcrafted entries in their initial lexicon. In addition, TBL outperforms the STC \cite{mairesse09}, HVS \cite{he05} and MLN \cite{meza08b} parsers. Concerning the TownInfo dataset, Table \ref{tbl:results-final} shows that TBL produces 87.82\% of F-measure, which represents a 3.28\% improvement over the handcrafted Phoenix parser, while being competitive with the STC model - TBL's performance is only 0.76\% lower.

% We believe that STC is performs better because it consider the STC classifier use all features at one time. STC makes decision in one step using all the features rather than making several decisions by several rules as STEP.
% We found that the dialogue act type recognition accuracy of the STEP parser is lower than STC's; as a result, we tried to use SVM as STC does to classify dialogue act types. 
% We hoped for an increase of F-measure as result of increased dialogue act type accuracy. However, we did not get any increase in F-measure.

% \efgr{fig:learning:curve}{The learning curve shows the relation between number of learned rules and the F-measure for both TI and ATIS corpora.}

% As is shown in the figure \ref{fig:learning:curve}, learning curves for both training data and development data are very steep. Although our current strategy for choosing the final number of rules for decoding is to keep only the rules for which we obtain highest F-measure on the development data, we could use much less rules without scarifying accuracy. For example, we accepted 0.1\% lower F-measure on the development data than we would need only YYY rules in comparison with XXX rules if select the number of rules based in the highest F-measure. In contrast, the initial lexicon the CCG parser \cite{zettlemoyer07} contains about 180 complex entries for general English words or phrases and yet additional lexical entries must be learned. \textbf{explain better}

Table \ref{tbl:results:contrast} shows a contrast between the full system and the system with no features extracted from dependency trees and the system with no locality constraints. Experiments were carried out on the ATIS development dataset. The results show that if the dependency tree features are removed or the locality constraints are not used, the performance degrades.

The learning time of the TBL parser is acceptable and the parsing process is efficient. 
First, the learning time is about 24 hours on an Intel Pentium 2.8GHz for each dataset. The TBL parser generates up to 1M potential transformation rules in each iteration; however, only a fraction of these rules have to be tested because the search space can be efficiently organised \cite{brill95}.
Second, the TBL parser is able to parse an utterance in 6ms while the STC parser needs 200ms on average \cite{mairesse09}. We cannot report on speed the other approaches because such information is not publicly available.

% The efficiency of the TBL parser results from a considerably low number of learnt rules.
The TBL parser is very efficient on domains such as ATIS and TownInfo because the final list of learnt rules is small.
% a number of operations needed to parse a utterance is limited because the number of learnt rules is low. 
There are 17 unique dialogue acts and 66 unique slots in the ATIS dataset and the total number of learnt rules is 372. This results in 4.5 rules per semantic concept on average. In the TownInfo dataset, we have 14 dialogue acts and 14 slots and the total number of learnt rules is 195. The average number of rules per semantic concept is 6.9. The number of semantic concepts per utterance is 5 on average.



% Lexical realizations of a slot can overlap with lexical realization of neighbouring slots. It is shows to be important pattern, for example in the trigram (city-0,and,city-1) is very common for utterance including ''between city-0 and city-1``. The lexical realizations city-0, city-1 respectively would be classified as from.city, and city-1 just because we know the  

\begin{table}
\begin{center}
\begin{tabular}{|l|ccc|}
\hline \makebox[2.99cm]{\bf Parser} & \makebox[0.8cm]{\bf Prec} & \makebox[0.8cm]{\bf Rec} & \bf F \\ \hline 
\multicolumn{4}{|l|}{\textbf{ATIS development dataset:}} \\
\hline
TBL   & 93.95 & 93.70 & 93.82 \\
No locality constraints & 93.38 & 92.64 & 93.01 \\
No dependency tree features  & 92.78 & 92.04 & 92.41 \\
\hline
\end{tabular}
\end{center}
\vspace{-0.5cm}
\caption{Comparison of different aspects of the TBL method on the ATIS development dataset.
% The second row presents results without using locality constraints described in section \ref{sec:locality:constrain}. The third row shows results of the method without features extracted from dependency trees described in section \ref{sec:dep:trees}. 
}
\label{tbl:results:contrast} 
\end{table}

\section{Conclusion} \label{sec:conlusion}

This paper presents a novel application of TBL for semantic parsing. Our method learns a sequence of rules which iteratively transforms the initial semantics into the correct semantics.
% It significantly differs from the method presented by Kate et al \cite{kate05} where they were rewriting utterances and replacing words with semantic concepts. 
The TBL parser was applied to two very different domains and it was shown that its performance is competitive with respect to the state-of-the-art semantic parsers on both datasets. 
% Results show that our method is competitive with respect to Zettlemoyer \& Collins' PCCG model \cite{zettlemoyer07} on ATIS dataset. In addition, 
On the ATIS dataset, TBL outperforms STC, HVS and MLN parsers by 1.27\%, 2.75\%, and 5.44\% respectively \cite{mairesse09,he05,meza08b}. We also show that TBL outperforms the handcrafted Phoenix parser by 3.28\% on ASR output of the TownInfo dataset \cite{mairesse09}.

Although the TBL approach cannot directly generate an N-best list of hypotheses with confidence scores, several methods have been developed to alleviate this problem. For example, transformation rules can be converted into decision trees from which informative probability distributions on the class labels can be obtained \cite{florian00}. In future work, we plan to investigate how to adapt the TBL method to obtain multiple hypotheses and confidence scores, and extend the model to richer domains where the ability to model long-range dependecies might be more important.

% \section{Acknowledgments}
% We would like to thank to Luke Zettlemoyer and I.V. Meza-Ruiz for valuable discussions about their methods.

\eightpt
\bibliographystyle{IEEEtran}
\bibliography{my}

\end{document}
