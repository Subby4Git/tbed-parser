%
% File naaclhlt2009.tex
%
% Contact: nasmith@cs.cmu.edu

\documentclass[11pt]{article}
\usepackage{naaclhlt2009}
\usepackage{times}
\usepackage{latexsym}
\setlength\titlebox{6.5cm}    % Expanding the titlebox

\title{Semantic Transformation-based Error-driven Parser}

\author{Filip Jurcicek, XXX \\
  Department of Engineering \\
  University of Cambridge \\
  {\tt fj228@cam.ac.uk, XXX}}

\date{}

\begin{document}
\maketitle
\begin{abstract}
  In this paper, we present a semantic parser which transforms initial naive semantic hypothesis into correct semantics by using a ordered set of rules. These rules are learnt automatically from the training corpus with no linguistic knowledge.
\end{abstract}

\section{Introduction}


\section{Related work}

Subsection~\ref{ssec:first}).

\section{Algorithm}

To limit overfiting the training data, we prune the rules which are learnt at the end of the learning. 

Although we use the 

% \begin{table}
% \begin{center}
% \begin{tabular}{|l|rl|}
% \hline \bf Type of Text & \bf Font Size & \bf Style \\ \hline
% paper title & 15 pt & bold \\
% author names & 12 pt & bold \\
% author affiliation & 12 pt & \\
% the word ``Abstract'' & 12 pt & bold \\
% section titles & 12 pt & bold \\
% document text & 11 pt  &\\
% abstract text & 10 pt & \\
% captions & 10 pt & \\
% bibliography & 10 pt & \\
% footnotes & 9 pt & \\
% \hline
% \end{tabular}
% \end{center}
% \caption{\label{font-table} Font guide. }
% \end{table}
% 
% {\bf Captions}: Provide a caption for every illustration; number each one
% sequentially in the form:  ``Figure 1. Caption of the Figure.'' ``Table 1.
% Caption of the Table.''  Type the captions of the figures and 
% tables below the body, using 10 point text.  

\section{Discussion}
The number of learnt rules is surprisingly small. As is shown in the figure \ref{fig:learning:curve}, learning curves for both training data and development data are very steep for both data corpora. Although our current strategy for choosing the final number of rules for decoding is to keep only the rules for which we obtain highest F-measure on the development data, we could use much less rules without scarifying accuracy. For example, we accepted 0.1\% lower F-measure on the development data than we would need only YYY rules in comparison with XXX rules if select the number of rules based in the highest F-measure.
In contrast, the initial lexicon the CCG parser \cite{zettlemoyer2007} contains about 180 sometimes very complex entries for general English and yet additional lexical entries must be learnt. 


Also, the number of rules per semantic concept (dialogue act or slot name) is very low. In TI data, we have XXX different dialogue acts and XXX slot and the average number of rules per semantic concept is XXX. In case of ATIS data, we have XXX dialogue acts and XXX slots and the average number of rules per semantic concept is XXX.

\section*{Acknowledgments}

Do not number the acknowledgment section.





















\begin{thebibliography}{}
\bibitem[\protect\citename{Aho and Ullman}1972]{Aho:72}
Alfred~V. Aho and Jeffrey~D. Ullman.
\newblock 1972.
\newblock {\em The Theory of Parsing, Translation and Compiling}, volume~1.
\newblock Prentice-{Hall}, Englewood Cliffs, NJ.

\bibitem[\protect\citename{{American Psychological Association}}1983]{APA:83}
{American Psychological Association}.
\newblock 1983.
\newblock {\em Publications Manual}.
\newblock American Psychological Association, Washington, DC.

\bibitem[\protect\citename{{Association for Computing Machinery}}1983]{ACM:83}
{Association for Computing Machinery}.
\newblock 1983.
\newblock {\em Computing Reviews}, 24(11):503--512.

\bibitem[\protect\citename{Chandra \bgroup et al.\egroup }1981]{Chandra:81}
Ashok~K. Chandra, Dexter~C. Kozen, and Larry~J. Stockmeyer.
\newblock 1981.
\newblock Alternation.
\newblock {\em Journal of the Association for Computing Machinery},
  28(1):114--133.

\bibitem[\protect\citename{Gusfield}1997]{Gusfield:97}
Dan Gusfield.
\newblock 1997.
\newblock {\em Algorithms on Strings, Trees and Sequences}.
\newblock Cambridge University Press, Cambridge, UK.

\end{thebibliography}

\end{document}
